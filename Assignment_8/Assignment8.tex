\documentclass[journal,12pt,twocolumn]{IEEEtran}

\usepackage{setspace}
\usepackage{gensymb}
\singlespacing
\usepackage[cmex10]{amsmath}
\usepackage{relsize}
\usepackage{amsthm}
\usepackage{newtxtext}
\usepackage{mathrsfs}
\usepackage{txfonts}
\usepackage{stfloats}
\usepackage{bm}
\usepackage{cite}
\usepackage{cases}
\usepackage{subfig}

\usepackage{longtable}
\usepackage{multirow}

\usepackage[inline]{enumitem}   
\makeatletter
% This command ignores the optional argument for itemize and enumerate lists
\newcommand{\inlineitem}[1][]{%
\ifnum\enit@type=\tw@
    {\descriptionlabel{#1}}
  \hspace{\labelsep}%
\else
  \ifnum\enit@type=\z@
       \refstepcounter{\@listctr}\fi
    \quad\@itemlabel\hspace{\labelsep}%
\fi}
\makeatother
\usepackage{esdiff}
\usepackage{mathtools}
\usepackage{steinmetz}
\usepackage{tikz}
\usepackage{circuitikz}
\usepackage{verbatim}
\usepackage{tfrupee}
\usepackage[breaklinks=true]{hyperref}
\usepackage{graphicx}
\usepackage{tkz-euclide}

\usetikzlibrary{calc,math}
\usepackage{listings}
    \usepackage{color}                                            %%
    \usepackage{array}                                            %%
    \usepackage{longtable}                                        %%
    \usepackage{calc}                                             %%
    \usepackage{multirow}                                         %%
    \usepackage{hhline}                                           %%
    \usepackage{ifthen}                                           %%
    \usepackage{lscape}     
\usepackage{multicol}
\usepackage{chngcntr}
\DeclareMathOperator*{\Res}{Res}

\renewcommand\thesection{\arabic{section}}
\renewcommand\thesubsection{\thesection.\arabic{subsection}}
\renewcommand\thesubsubsection{\thesubsection.\arabic{subsubsection}}

\renewcommand\thesectiondis{\arabic{section}}
\renewcommand\thesubsectiondis{\thesectiondis.\arabic{subsection}}
\renewcommand\thesubsubsectiondis{\thesubsectiondis.\arabic{subsubsection}}


\hyphenation{op-tical net-works semi-conduc-tor}
\def\inputGnumericTable{}                                 %%

\lstset{
%language=C,
frame=single, 
breaklines=true,
columns=fullflexible
}
\begin{document}
\newcommand{\BEQA}{\begin{eqnarray}}
\newcommand{\EEQA}{\end{eqnarray}}
\newcommand{\define}{\stackrel{\triangle}{=}}
\bibliographystyle{IEEEtran}
\raggedbottom
\setlength{\parindent}{0pt}
\providecommand{\mbf}{\mathbf}
\providecommand{\pr}[1]{\ensuremath{\Pr\left(#1\right)}}
\providecommand{\qfunc}[1]{\ensuremath{Q\left(#1\right)}}
\providecommand{\sbrak}[1]{\ensuremath{{}\left[#1\right]}}
\providecommand{\lsbrak}[1]{\ensuremath{{}\left[#1\right.}}
\providecommand{\rsbrak}[1]{\ensuremath{{}\left.#1\right]}}
\providecommand{\brak}[1]{\ensuremath{\left(#1\right)}}
\providecommand{\lbrak}[1]{\ensuremath{\left(#1\right.}}
\providecommand{\rbrak}[1]{\ensuremath{\left.#1\right)}}
\providecommand{\cbrak}[1]{\ensuremath{\left\{#1\right\}}}
\providecommand{\lcbrak}[1]{\ensuremath{\left\{#1\right.}}
\providecommand{\rcbrak}[1]{\ensuremath{\left.#1\right\}}}
\theoremstyle{remark}
\newtheorem{rem}{Remark}
\newcommand{\sgn}{\mathop{\mathrm{sgn}}}
\providecommand{\abs}[1]{\vert#1\vert}
\providecommand{\res}[1]{\Res\displaylimits_{#1}} 
\providecommand{\norm}[1]{\lVert#1\rVert}
%\providecommand{\norm}[1]{\lVert#1\rVert}
\providecommand{\mtx}[1]{\mathbf{#1}}
\providecommand{\mean}[1]{E[ #1 ]}
\providecommand{\fourier}{\overset{\mathcal{F}}{ \rightleftharpoons}}
%\providecommand{\hilbert}{\overset{\mathcal{H}}{ \rightleftharpoons}}
\providecommand{\system}{\overset{\mathcal{H}}{ \longleftrightarrow}}
  %\newcommand{\solution}[2]{\textbf{Solution:}{#1}}
\newcommand{\solution}{\noindent \textbf{Solution: }}
\newcommand{\cosec}{\,\text{cosec}\,}
\providecommand{\dec}[2]{\ensuremath{\overset{#1}{\underset{#2}{\gtrless}}}}
\newcommand{\myvec}[1]{\ensuremath{\begin{pmatrix}#1\end{pmatrix}}}
\newcommand{\mydet}[1]{\ensuremath{\begin{vmatrix}#1\end{vmatrix}}}
\numberwithin{equation}{subsection}
\makeatletter
\@addtoreset{figure}{problem}
\makeatother
\let\StandardTheFigure\thefigure
\let\vec\mathbf
\renewcommand{\thefigure}{\theproblem}
\def\putbox#1#2#3{\makebox[0in][l]{\makebox[#1][l]{}\raisebox{\baselineskip}[0in][0in]{\raisebox{#2}[0in][0in]{#3}}}}
     \def\rightbox#1{\makebox[0in][r]{#1}}
     \def\centbox#1{\makebox[0in]{#1}}
     \def\topbox#1{\raisebox{-\baselineskip}[0in][0in]{#1}}
     \def\midbox#1{\raisebox{-0.5\baselineskip}[0in][0in]{#1}}
\vspace{3cm}
\title{Assignment 8}
\author{Gorantla Pranav Sai- CS20BTECH11018}
\maketitle
\newpage
\bigskip
\renewcommand{\thefigure}{\theenumi}
\renewcommand{\thetable}{\theenumi}
Download all python codes from 
\begin{lstlisting}
https://github.com/pranav-159/ai1103_Probability_and_Random_variables/blob/main/Assignment_8/codes/experimental_verification_Assignment8.py
\end{lstlisting}
\section{Problem}
\textbf{GATE 2021 (ME-SET1), Q.42 ( ME section)}\\  
   Consider a single machine workstation to which jobs arrive according to a
Poisson distribution with a mean arrival rate of 12 jobs/hour. The process
time of the workstation is exponentially distributed with a mean of 4
minutes. The expected number of jobs at the workstation at any given
point of time is \ldots (\textit{round off to the nearest integer}).
\section{Solution}
In a Poisson process,
 \begin{align}
          \pr{X=x}&= e^{-\lambda \Delta t} \frac{(\lambda \Delta t)^{x}}{x\,!}
 \end{align}
If $\Delta t \rightarrow 0$ then probability of having only one Poisson job is  
  \begin{align}
         \pr{X=1}=\lambda \Delta t \label{singlejob-condition}
 \end{align}
 Some assumptions:\\
 In time interval $\Delta t$,
 \begin{itemize}
     \item Exactly one job is arrived  
     \item or Exactly one job is completed
     \item or Nothing happens
 \end{itemize}
Assumptions seem quite reasonable as $\Delta t$ is very small then the probability of occurrence of more than one poisson job is very low.\\
 For job arrival,
 \begin{itemize}
 \item It is distributed according to Poisson distribution.
     \item Its
 Rate parameter $\lambda $=12 jobs/hour.
 \item Using \eqref{singlejob-condition},Probability that a single job arrives in a small interval $\Delta t=\lambda\Delta t$.
 \end{itemize}
 For Job completions,
 \begin{itemize}
     \item  Job completion time is distributed exponentially with mean of 4 minutes 
     \item Then we can assume that no. of job completions are distributed as Poisson distribution with rate parameter $\mu$ = 15 jobs/hour
     \item Once again using \eqref{singlejob-condition},
 Probability that a single job will be completed in a small interval $\Delta t=\mu \Delta t$
 \end{itemize}
 Some notations,
 \begin{table}[h]
\begin{tabular}{|c|p{6cm}|}
\hline
\textbf{Parameter} & \textbf{Definition}                               \\ \hline
$\lambda$          & Poisson rate parameter for the arrival of jobs    \\ \hline
$\mu$              & Poisson rate parameter for the completion of jobs \\ \hline
$\lambda \Delta t$ & Probability that a single job arrives in a small interval $\Delta t$\\\hline 
$\mu \Delta t$ & Probability that a single job will be completed in a small interval $\Delta t$\\\hline 
$P_j(t)$             & probability of having j jobs at workstation at time t \\\hline
$\pi_j$            & steady probability of having j jobs at workstation\\\hline
\end{tabular}
\caption{Parameters and their definitions used in the problem}
\label{tab:parameters}
\end{table}
 \begin{itemize}
     \item  Initial no.of jobs at workstation is 0.
     \item Let $P_{j}(t)$ denote the probability of having $j$ jobs waiting at the workstation at the time $t$ for this initial case.
     \item After a long time,probability of having  j jobs becomes steady.
     \item Let us denote steady state probability of having j jobs as $\pi_j$.
 \end{itemize}
 Condition which ensures that steady state is reached is
 \begin{align}
     \diff{P_j(t)}{t}&=0\\
     \lim_{\Delta t\rightarrow 0}\dfrac{P_j(t+\Delta t)-P_j (t)}{\Delta t}&=0\label{steady-condition}
 \end{align}
  We can reach a state of $j$ jobs at time $t+\Delta t$ from
  \begin{itemize}
      \item A state of $j-1$ jobs at time $t$ with a new job arriving in the next $\Delta t$
      \item A state of $j+1$ jobs at time $t$ with a job completing in the next $\Delta t$
      \item A state of $j$ jobs at time $t$ and nothing happening in the next $\Delta t$
  \end{itemize}
 Assuming time  $t$ is long enough for the occurrence of steady state.The above relations can be shown in probability equations as:  
 \begin{align}
     P_j(t+\Delta t)&=P_{j-1}(t)\lambda\Delta t+ P_{j+1}(t)\mu \Delta t \nonumber\\&+P_j (t) (1-\lambda\Delta t -\mu \Delta t)\\
     \dfrac{P_j(t+\Delta t)-P_j (t)}{\Delta t}&= P_{j-1}(t)\lambda +P_{j+1}(t)\mu\nonumber\\& - P_j(t)\lambda -P_j(t)\mu
\end{align}
Using \eqref{steady-condition} we get,
\begin{align}
     \implies P_{j-1}(t)\lambda +P_{j+1}(t)\mu&=P_j(t)\lambda +P_j(t)\mu \\
     \pi_{j-1}\lambda +\pi_{j+1}\mu&=\pi_j\lambda +\pi_j\mu\label{recursive} 
 \end{align}
 Note that the above equations are  for $j \geq 1$. \\
 For j=0 jobs at time $t+\Delta t$ we can reach it from j=1 job at time $t$ with a job completion in the next $\Delta t$ or else stay at j=0 at time $t$ and do nothing the next $\Delta t$
\begin{align}
    P_0(t+\Delta t)&=P_1(t)\mu \Delta t+\nonumber\\&~~~~P_0(t)(1-\lambda \Delta t)\\
    \dfrac{P_0(t+\Delta t)-P_0(t)}{\Delta t}&=P_1(t) \mu \Delta t-P_0(t)\lambda \Delta t
\end{align}
Once again using \eqref{steady-condition},we will get,
\begin{align}
    P_0(t)\lambda \Delta t&= P_1(t) \mu \Delta t\\
    P_0(t)\lambda&=P_1(t) \mu \\
    \pi_0 \lambda&=\pi_1\mu\label{base}
\end{align}
Solving \eqref{base} and \eqref{recursive} with appropriate j one by one,we will get $P_j$ in terms of $P_0$ as
\begin{equation}
    P_j=\brak{\dfrac{\lambda}{\mu}}^jP_0 
\end{equation}
consider $\rho = \dfrac{\lambda}{\mu}$.
\begin{table}[h]
\begin{tabular}{|c|p{6cm}|}
\hline
\textbf{Parameter} & \textbf{Definition}                               \\ \hline
$E(j)$             & Expected no. of jobs at workstation \\ \hline
$\rho$             & $\dfrac{\lambda}{\mu}$\\\hline
\end{tabular}
\caption{Parameters and their definitions used in the problem}
\label{tab:parameters}
\end{table}
\begin{equation}
    P_j=\rho^j P_0 \label{solution}
\end{equation}
We can prove that \eqref{solution} is indeed the solution of recursion equation \eqref{recursive} by using mathematical induction.\\
Assuming $\rho<1$,let us calculate $P_0$ in terms of $\rho$
\begin{align}
    \sum_{j=0}^{\infty}P_j&=1\\
    \sum_{j=0}^{\infty}\rho^j P_0 &=1\\
    \dfrac{P_0}{1-\rho}&=1\\
    P_0&=1-\rho
\end{align}
This yields,\\
\begin{equation}
    P_j=\rho^j(1-\rho)
\end{equation}
Let us calculate expected value of jobs waiting at workstation.
\begin{align}
    E(j)&=\sum_{j=0}^{\infty}jP_j\\
    E(j)&=(1-\rho)\sum_{j=0}^{\infty}j\rho^j\label{first}\\
    \rho E(j)&=(1-\rho)\sum_{j=0}^{\infty}j\rho^{j+1}\\
    \rho E(j)&=(1-\rho)\sum_{j=1}^{\infty}(j-1)\rho^{j}\label{second}
\end{align}
Subtracting \eqref{second} from \eqref{first}.we get,
\begin{align}
    (1-\rho)E(j)&=(1-\rho)\sum_{j=1}^{\infty}\rho^j\\
    E(j)&=\sum_{j=1}^{\infty}\rho^j\\
    E(j)&=\dfrac{\rho}{1-\rho}\label{expect}
\end{align}
In our case $\rho=\dfrac{\lambda}{\mu}=\dfrac{12}{15}=\dfrac{4}{5}$.\\\\Substituting it in the \eqref{expect} we get,\\
\begin{equation}
    E(j)=4
\end{equation}
$\therefore$ Expected no.of jobs at workstation is 4.
\end{document}    