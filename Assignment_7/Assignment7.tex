\documentclass[journal,12pt,twocolumn]{IEEEtran}
\newcommand\hmmax{0}
\newcommand\bmmax{0}
\usepackage{dsfont}
\usepackage{setspace}
\usepackage{gensymb}
\singlespacing
\usepackage[cmex10]{amsmath}
\usepackage{relsize}
\usepackage{amsthm}
\usepackage{newtxtext}
\usepackage{mathrsfs}
\usepackage{txfonts}
\usepackage{stfloats}
\usepackage{bm}
\usepackage{cite}
\usepackage{cases}
\usepackage{subfig}

\usepackage{longtable}
\usepackage{multirow}

\usepackage[inline]{enumitem}   
\makeatletter
% This command ignores the optional argument for itemize and enumerate lists
\newcommand{\inlineitem}[1][]{%
\ifnum\enit@type=\tw@
    {\descriptionlabel{#1}}
  \hspace{\labelsep}%
\else
  \ifnum\enit@type=\z@
       \refstepcounter{\@listctr}\fi
    \quad\@itemlabel\hspace{\labelsep}%
\fi}
\makeatother
\usepackage{mathtools}
\usepackage{steinmetz}
\usepackage{tikz}
\usepackage{circuitikz}
\usepackage{verbatim}
\usepackage{tfrupee}
\usepackage[breaklinks=true]{hyperref}
\usepackage{graphicx}
\usepackage{tkz-euclide}

\usetikzlibrary{calc,math}
\usepackage{listings}
    \usepackage{color}                                            %%
    \usepackage{array}                                            %%
    \usepackage{longtable}                                        %%
    \usepackage{calc}                                             %%
    \usepackage{multirow}                                         %%
    \usepackage{hhline}                                           %%
    \usepackage{ifthen}                                           %%
    \usepackage{lscape}     
\usepackage{multicol}
\usepackage{chngcntr}
\DeclareMathOperator*{\Res}{Res}

\renewcommand\thesection{\arabic{section}}
\renewcommand\thesubsection{\thesection.\arabic{subsection}}
\renewcommand\thesubsubsection{\thesubsection.\arabic{subsubsection}}

\renewcommand\thesectiondis{\arabic{section}}
\renewcommand\thesubsectiondis{\thesectiondis.\arabic{subsection}}
\renewcommand\thesubsubsectiondis{\thesubsectiondis.\arabic{subsubsection}}


\hyphenation{op-tical net-works semi-conduc-tor}
\def\inputGnumericTable{}                                 %%

\lstset{
%language=C,
frame=single, 
breaklines=true,
columns=fullflexible
}
\begin{document}

\newcommand{\BEQA}{\begin{eqnarray}}
\newcommand{\EEQA}{\end{eqnarray}}
\newcommand{\define}{\stackrel{\triangle}{=}}
\bibliographystyle{IEEEtran}
\raggedbottom
\setlength{\parindent}{0pt}
\providecommand{\mbf}{\mathbf}
\providecommand{\pr}[1]{\ensuremath{\Pr\left(#1\right)}}
\providecommand{\qfunc}[1]{\ensuremath{Q\left(#1\right)}}
\providecommand{\sbrak}[1]{\ensuremath{{}\left[#1\right]}}
\providecommand{\lsbrak}[1]{\ensuremath{{}\left[#1\right.}}
\providecommand{\rsbrak}[1]{\ensuremath{{}\left.#1\right]}}
\providecommand{\brak}[1]{\ensuremath{\left(#1\right)}}
\providecommand{\lbrak}[1]{\ensuremath{\left(#1\right.}}
\providecommand{\rbrak}[1]{\ensuremath{\left.#1\right)}}
\providecommand{\cbrak}[1]{\ensuremath{\left\{#1\right\}}}
\providecommand{\lcbrak}[1]{\ensuremath{\left\{#1\right.}}
\providecommand{\rcbrak}[1]{\ensuremath{\left.#1\right\}}}
\theoremstyle{remark}
\newtheorem{rem}{Remark}
\newcommand{\sgn}{\mathop{\mathrm{sgn}}}
\providecommand{\abs}[1]{\vert#1\vert}
\providecommand{\res}[1]{\Res\displaylimits_{#1}} 
\providecommand{\norm}[1]{\lVert#1\rVert}
%\providecommand{\norm}[1]{\lVert#1\rVert}
\providecommand{\mtx}[1]{\mathbf{#1}}
\providecommand{\mean}[1]{E[ #1 ]}
\providecommand{\fourier}{\overset{\mathcal{F}}{ \rightleftharpoons}}
%\providecommand{\hilbert}{\overset{\mathcal{H}}{ \rightleftharpoons}}
\providecommand{\system}{\overset{\mathcal{H}}{ \longleftrightarrow}}
  %\newcommand{\solution}[2]{\textbf{Solution:}{#1}}
\newcommand{\solution}{\noindent \textbf{Solution: }}
\newcommand{\cosec}{\,\text{cosec}\,}
\providecommand{\dec}[2]{\ensuremath{\overset{#1}{\underset{#2}{\gtrless}}}}
\newcommand{\myvec}[1]{\ensuremath{\begin{pmatrix}#1\end{pmatrix}}}
\newcommand{\mydet}[1]{\ensuremath{\begin{vmatrix}#1\end{vmatrix}}}
\numberwithin{equation}{subsection}
\makeatletter
\@addtoreset{figure}{problem}
\makeatother
\let\StandardTheFigure\thefigure
\let\vec\mathbf
\renewcommand{\thefigure}{\theproblem}
\def\putbox#1#2#3{\makebox[0in][l]{\makebox[#1][l]{}\raisebox{\baselineskip}[0in][0in]{\raisebox{#2}[0in][0in]{#3}}}}
     \def\rightbox#1{\makebox[0in][r]{#1}}
     \def\centbox#1{\makebox[0in]{#1}}
     \def\topbox#1{\raisebox{-\baselineskip}[0in][0in]{#1}}
     \def\midbox#1{\raisebox{-0.5\baselineskip}[0in][0in]{#1}}
\vspace{3cm}
\title{Assignment 7}
\author{Gorantla Pranav Sai- CS20BTECH11018}
\maketitle
\newpage
\bigskip
\renewcommand{\thefigure}{\theenumi}
\renewcommand{\thetable}{\theenumi}

\section{Problem}
\textbf{GATE 2019 (ST) , Q.49 (Statistics section)}\\
   Let X be a random variable with characteristic function $\phi_X(\cdot)$ such that $\phi_X(2 \pi)=1$.Let $\mathds{Z}$ denote the set of integers.Then $P(X \in \mathds{Z})$ is equal to \ldots
\section{Solution}
  General solution for the characteristic solution which is consistent with our condition is,
  \begin{align}
      \phi_X(t)=\sum_{z=-\infty}^{\infty}\alpha_ze^{itz}& \text{~~where~~} z \in \mathds{Z} \text{~~and}\\
      &\sum_{z=-\infty}^{\infty}\alpha_z=1 \label{main condition}
  \end{align}
  Using Gil-Pelaez formula for probability density function,
  \begin{align}
      f_X(x)&=\dfrac{1}{4\pi}\int_{-\infty}^\infty e^{itx}\phi(-t)+e^{-itx}\phi(t)~dt\\
      &=\dfrac{1}{4\pi}\int_{-\infty}^\infty e^{itx}\sum_{z=-\infty}^{\infty}\alpha_ze^{itz}\nonumber\\&~~~~~~~~~+\dfrac{1}{4\pi}e^{-itx}\sum_{z=-\infty}^{\infty}\alpha_ze^{itz}~dt\\
      &=\dfrac{1}{4\pi}\sum_{z=-\infty}^{\infty}\int_{-\infty}^\infty\alpha_ze^{-it(x-z)}dt\nonumber\\&~~~~~~~~+\dfrac{1}{4\pi}\sum_{z=-\infty}^{\infty}\int_{-\infty}^\infty\alpha_ze^{it(x+z)}dt
\end{align}
We know that,
\begin{equation}
    \int_{-\infty}^{\infty}e^{\pm ik(x-x_{0})}dk=2\pi~\delta(x-x_{0}) \label{delta function}
\end{equation}
Using \eqref{delta function} we get,
\begin{align}
     f_X(x) &=\dfrac{1}{2}\sum_{z=-\infty}^{\infty}\alpha_z \delta(x-z)+\dfrac{1}{2}\sum_{z=-\infty}^{\infty}\alpha_z \delta(x+z)\\
     &=\dfrac{1}{2}\sum_{z=-\infty}^{\infty}(\alpha_z+\alpha_{-z})\delta(x-z)
\end{align}
  Taking $\beta_{|z|}=\dfrac{\alpha_z+\alpha_{-z}}{2}$,
  \begin{align}
       f_X(x)&=\sum_{z=-\infty}^{\infty}\beta_{|z|}\delta(x-z)
  \end{align}
  We know that,
  \begin{align}
      \pr{X=z_0|z_0 \in \mathds{Z}}&=\int_{-\infty}^{\infty} f_X(z_0)~dx\\
      &=\int_{-\infty}^{\infty}\sum_{z=-\infty}^{\infty}\beta_{|z|}\delta(z_0-z)~dx\\
      &=\beta_{|z_0|}
  \end{align}
  \begin{align}
      P(X \in \mathds{Z})&=\sum_{z_0=-\infty}^{\infty}\pr{X=z_0}\nonumber\\&~~~~~~~~\text{where~~} z_0 \in \mathds{Z}\\
      &=\sum_{z_0=-\infty}^{\infty}\beta_{|z_0|}\\
      &=\sum_{z_0=-\infty}^{\infty}\dfrac{\alpha_{z_0}+\alpha_{-z_0}}{2}\\
      &=\dfrac{\mathlarger{\sum}_{z_0=-\infty}^{\infty}\alpha_{z_0}+\mathlarger{\sum}_{z_0=-\infty}^{\infty}\alpha_{-z_0}}{2}
  \end{align}
  Using \eqref{main condition} we get,
  $$ P(X \in \mathds{Z})=1$$
\end{document}    